\documentclass[]{article}
\usepackage{lmodern}
\usepackage{amssymb,amsmath}
\usepackage{ifxetex,ifluatex}
\usepackage{fixltx2e} % provides \textsubscript
\ifnum 0\ifxetex 1\fi\ifluatex 1\fi=0 % if pdftex
  \usepackage[T1]{fontenc}
  \usepackage[utf8]{inputenc}
\else % if luatex or xelatex
  \ifxetex
    \usepackage{mathspec}
  \else
    \usepackage{fontspec}
  \fi
  \defaultfontfeatures{Ligatures=TeX,Scale=MatchLowercase}
\fi
% use upquote if available, for straight quotes in verbatim environments
\IfFileExists{upquote.sty}{\usepackage{upquote}}{}
% use microtype if available
\IfFileExists{microtype.sty}{%
\usepackage{microtype}
\UseMicrotypeSet[protrusion]{basicmath} % disable protrusion for tt fonts
}{}
\usepackage[margin=1in]{geometry}
\usepackage{hyperref}
\hypersetup{unicode=true,
            pdftitle={R Notebook},
            pdfborder={0 0 0},
            breaklinks=true}
\urlstyle{same}  % don't use monospace font for urls
\usepackage{color}
\usepackage{fancyvrb}
\newcommand{\VerbBar}{|}
\newcommand{\VERB}{\Verb[commandchars=\\\{\}]}
\DefineVerbatimEnvironment{Highlighting}{Verbatim}{commandchars=\\\{\}}
% Add ',fontsize=\small' for more characters per line
\usepackage{framed}
\definecolor{shadecolor}{RGB}{248,248,248}
\newenvironment{Shaded}{\begin{snugshade}}{\end{snugshade}}
\newcommand{\KeywordTok}[1]{\textcolor[rgb]{0.13,0.29,0.53}{\textbf{#1}}}
\newcommand{\DataTypeTok}[1]{\textcolor[rgb]{0.13,0.29,0.53}{#1}}
\newcommand{\DecValTok}[1]{\textcolor[rgb]{0.00,0.00,0.81}{#1}}
\newcommand{\BaseNTok}[1]{\textcolor[rgb]{0.00,0.00,0.81}{#1}}
\newcommand{\FloatTok}[1]{\textcolor[rgb]{0.00,0.00,0.81}{#1}}
\newcommand{\ConstantTok}[1]{\textcolor[rgb]{0.00,0.00,0.00}{#1}}
\newcommand{\CharTok}[1]{\textcolor[rgb]{0.31,0.60,0.02}{#1}}
\newcommand{\SpecialCharTok}[1]{\textcolor[rgb]{0.00,0.00,0.00}{#1}}
\newcommand{\StringTok}[1]{\textcolor[rgb]{0.31,0.60,0.02}{#1}}
\newcommand{\VerbatimStringTok}[1]{\textcolor[rgb]{0.31,0.60,0.02}{#1}}
\newcommand{\SpecialStringTok}[1]{\textcolor[rgb]{0.31,0.60,0.02}{#1}}
\newcommand{\ImportTok}[1]{#1}
\newcommand{\CommentTok}[1]{\textcolor[rgb]{0.56,0.35,0.01}{\textit{#1}}}
\newcommand{\DocumentationTok}[1]{\textcolor[rgb]{0.56,0.35,0.01}{\textbf{\textit{#1}}}}
\newcommand{\AnnotationTok}[1]{\textcolor[rgb]{0.56,0.35,0.01}{\textbf{\textit{#1}}}}
\newcommand{\CommentVarTok}[1]{\textcolor[rgb]{0.56,0.35,0.01}{\textbf{\textit{#1}}}}
\newcommand{\OtherTok}[1]{\textcolor[rgb]{0.56,0.35,0.01}{#1}}
\newcommand{\FunctionTok}[1]{\textcolor[rgb]{0.00,0.00,0.00}{#1}}
\newcommand{\VariableTok}[1]{\textcolor[rgb]{0.00,0.00,0.00}{#1}}
\newcommand{\ControlFlowTok}[1]{\textcolor[rgb]{0.13,0.29,0.53}{\textbf{#1}}}
\newcommand{\OperatorTok}[1]{\textcolor[rgb]{0.81,0.36,0.00}{\textbf{#1}}}
\newcommand{\BuiltInTok}[1]{#1}
\newcommand{\ExtensionTok}[1]{#1}
\newcommand{\PreprocessorTok}[1]{\textcolor[rgb]{0.56,0.35,0.01}{\textit{#1}}}
\newcommand{\AttributeTok}[1]{\textcolor[rgb]{0.77,0.63,0.00}{#1}}
\newcommand{\RegionMarkerTok}[1]{#1}
\newcommand{\InformationTok}[1]{\textcolor[rgb]{0.56,0.35,0.01}{\textbf{\textit{#1}}}}
\newcommand{\WarningTok}[1]{\textcolor[rgb]{0.56,0.35,0.01}{\textbf{\textit{#1}}}}
\newcommand{\AlertTok}[1]{\textcolor[rgb]{0.94,0.16,0.16}{#1}}
\newcommand{\ErrorTok}[1]{\textcolor[rgb]{0.64,0.00,0.00}{\textbf{#1}}}
\newcommand{\NormalTok}[1]{#1}
\usepackage{graphicx,grffile}
\makeatletter
\def\maxwidth{\ifdim\Gin@nat@width>\linewidth\linewidth\else\Gin@nat@width\fi}
\def\maxheight{\ifdim\Gin@nat@height>\textheight\textheight\else\Gin@nat@height\fi}
\makeatother
% Scale images if necessary, so that they will not overflow the page
% margins by default, and it is still possible to overwrite the defaults
% using explicit options in \includegraphics[width, height, ...]{}
\setkeys{Gin}{width=\maxwidth,height=\maxheight,keepaspectratio}
\IfFileExists{parskip.sty}{%
\usepackage{parskip}
}{% else
\setlength{\parindent}{0pt}
\setlength{\parskip}{6pt plus 2pt minus 1pt}
}
\setlength{\emergencystretch}{3em}  % prevent overfull lines
\providecommand{\tightlist}{%
  \setlength{\itemsep}{0pt}\setlength{\parskip}{0pt}}
\setcounter{secnumdepth}{0}
% Redefines (sub)paragraphs to behave more like sections
\ifx\paragraph\undefined\else
\let\oldparagraph\paragraph
\renewcommand{\paragraph}[1]{\oldparagraph{#1}\mbox{}}
\fi
\ifx\subparagraph\undefined\else
\let\oldsubparagraph\subparagraph
\renewcommand{\subparagraph}[1]{\oldsubparagraph{#1}\mbox{}}
\fi

%%% Use protect on footnotes to avoid problems with footnotes in titles
\let\rmarkdownfootnote\footnote%
\def\footnote{\protect\rmarkdownfootnote}

%%% Change title format to be more compact
\usepackage{titling}

% Create subtitle command for use in maketitle
\newcommand{\subtitle}[1]{
  \posttitle{
    \begin{center}\large#1\end{center}
    }
}

\setlength{\droptitle}{-2em}

  \title{R Notebook}
    \pretitle{\vspace{\droptitle}\centering\huge}
  \posttitle{\par}
    \author{}
    \preauthor{}\postauthor{}
    \date{}
    \predate{}\postdate{}
  

\begin{document}
\maketitle

\section{Problem 1}\label{problem-1}

\begin{enumerate}
\def\labelenumi{\alph{enumi})}
\tightlist
\item
  No
\item
  Yes: Classification
\item
  Yes: Regression
\item
  No
\item
  Yes: Regression
\item
  Yes: Classification (if we are trying to predict if a given user will
  click)
\item
  Yes: Anomaly detection
\item
  No
\end{enumerate}

\section{Problem 2}\label{problem-2}

Continuous, Quantitative, Ratio (Assuming that 4 is twice as bright as
2) Continuous, Quantitative, Ratio Discrete, Qualitative, Ordinal
Continuous, Quantitative, Interval (We can consider the 5 hour interval
between 10AM and 3PM) Discrete, Qualitative Ordinal

\section{Problem 3}\label{problem-3}

\begin{enumerate}
\def\labelenumi{\alph{enumi})}
\item
\end{enumerate}

\begin{Shaded}
\begin{Highlighting}[]
\KeywordTok{library}\NormalTok{(datasets)}
\KeywordTok{data}\NormalTok{(iris)}
\KeywordTok{sprintf}\NormalTok{(}\StringTok{"Number of datapoints: %d"}\NormalTok{, }\KeywordTok{nrow}\NormalTok{(iris))}
\end{Highlighting}
\end{Shaded}

\begin{verbatim}
## [1] "Number of datapoints: 150"
\end{verbatim}

\begin{Shaded}
\begin{Highlighting}[]
\KeywordTok{sprintf}\NormalTok{(}\StringTok{"Number of features: %d"}\NormalTok{, }\KeywordTok{ncol}\NormalTok{(iris))}
\end{Highlighting}
\end{Shaded}

\begin{verbatim}
## [1] "Number of features: 5"
\end{verbatim}

\begin{Shaded}
\begin{Highlighting}[]
\KeywordTok{sprintf}\NormalTok{(}\StringTok{"Number of classes: %d"}\NormalTok{, }\KeywordTok{length}\NormalTok{(}\KeywordTok{unique}\NormalTok{(iris}\OperatorTok{$}\NormalTok{Species)))}
\end{Highlighting}
\end{Shaded}

\begin{verbatim}
## [1] "Number of classes: 3"
\end{verbatim}

\begin{enumerate}
\def\labelenumi{\alph{enumi})}
\setcounter{enumi}{1}
\item
\end{enumerate}

\begin{Shaded}
\begin{Highlighting}[]
\KeywordTok{summary}\NormalTok{(iris)}
\end{Highlighting}
\end{Shaded}

\begin{verbatim}
##   Sepal.Length    Sepal.Width     Petal.Length    Petal.Width   
##  Min.   :4.300   Min.   :2.000   Min.   :1.000   Min.   :0.100  
##  1st Qu.:5.100   1st Qu.:2.800   1st Qu.:1.600   1st Qu.:0.300  
##  Median :5.800   Median :3.000   Median :4.350   Median :1.300  
##  Mean   :5.843   Mean   :3.057   Mean   :3.758   Mean   :1.199  
##  3rd Qu.:6.400   3rd Qu.:3.300   3rd Qu.:5.100   3rd Qu.:1.800  
##  Max.   :7.900   Max.   :4.400   Max.   :6.900   Max.   :2.500  
##        Species  
##  setosa    :50  
##  versicolor:50  
##  virginica :50  
##                 
##                 
## 
\end{verbatim}

\begin{Shaded}
\begin{Highlighting}[]
\KeywordTok{sprintf}\NormalTok{(}\StringTok{"Sepal.Length Std.Dev.: %f"}\NormalTok{, }\KeywordTok{sd}\NormalTok{(iris}\OperatorTok{$}\NormalTok{Sepal.Length))}
\end{Highlighting}
\end{Shaded}

\begin{verbatim}
## [1] "Sepal.Length Std.Dev.: 0.828066"
\end{verbatim}

\begin{Shaded}
\begin{Highlighting}[]
\KeywordTok{sprintf}\NormalTok{(}\StringTok{"Sepal.Width Std.Dev.: %f"}\NormalTok{, }\KeywordTok{sd}\NormalTok{(iris}\OperatorTok{$}\NormalTok{Sepal.Width))}
\end{Highlighting}
\end{Shaded}

\begin{verbatim}
## [1] "Sepal.Width Std.Dev.: 0.435866"
\end{verbatim}

\begin{Shaded}
\begin{Highlighting}[]
\KeywordTok{sprintf}\NormalTok{(}\StringTok{"Petal.Length Std.Dev.: %f"}\NormalTok{, }\KeywordTok{sd}\NormalTok{(iris}\OperatorTok{$}\NormalTok{Petal.Length))}
\end{Highlighting}
\end{Shaded}

\begin{verbatim}
## [1] "Petal.Length Std.Dev.: 1.765298"
\end{verbatim}

\begin{Shaded}
\begin{Highlighting}[]
\KeywordTok{sprintf}\NormalTok{(}\StringTok{"Petal.Width Std.Dev.: %f"}\NormalTok{, }\KeywordTok{sd}\NormalTok{(iris}\OperatorTok{$}\NormalTok{Petal.Width))}
\end{Highlighting}
\end{Shaded}

\begin{verbatim}
## [1] "Petal.Width Std.Dev.: 0.762238"
\end{verbatim}

\begin{enumerate}
\def\labelenumi{\alph{enumi})}
\setcounter{enumi}{2}
\item
\end{enumerate}

\begin{Shaded}
\begin{Highlighting}[]
\KeywordTok{plot}\NormalTok{(iris}\OperatorTok{$}\NormalTok{Sepal.Length }\OperatorTok{~}\StringTok{ }\NormalTok{iris}\OperatorTok{$}\NormalTok{Petal.Length,}
      \DataTypeTok{xlab =} \StringTok{"Petal Length (cm)"}\NormalTok{,}
      \DataTypeTok{ylab =} \StringTok{"Sepal Length (cm)"}\NormalTok{,}
      \DataTypeTok{pch =} \KeywordTok{c}\NormalTok{(}\DecValTok{16}\NormalTok{, }\DecValTok{17}\NormalTok{, }\DecValTok{18}\NormalTok{)[}\KeywordTok{unclass}\NormalTok{(iris}\OperatorTok{$}\NormalTok{Species)], }
      \DataTypeTok{main =} \StringTok{"Iris Dataset"}\NormalTok{,}
      \DataTypeTok{col =} \KeywordTok{c}\NormalTok{(}\StringTok{"red"}\NormalTok{, }\StringTok{"green"}\NormalTok{,}\StringTok{"blue"}\NormalTok{)[}\KeywordTok{unclass}\NormalTok{(iris}\OperatorTok{$}\NormalTok{Species)],}
      \DataTypeTok{data =}\NormalTok{ iris)}

\KeywordTok{legend}\NormalTok{(}\StringTok{"topleft"}\NormalTok{, }
        \DataTypeTok{legend =} \KeywordTok{c}\NormalTok{(}\StringTok{"setosa"}\NormalTok{, }\StringTok{"versicolor"}\NormalTok{, }\StringTok{"virginica"}\NormalTok{), }
        \DataTypeTok{text.col =} \KeywordTok{c}\NormalTok{(}\StringTok{"red"}\NormalTok{, }\StringTok{"green"}\NormalTok{,}\StringTok{"blue"}\NormalTok{),}
        \DataTypeTok{col =} \KeywordTok{c}\NormalTok{(}\StringTok{"red"}\NormalTok{, }\StringTok{"green"}\NormalTok{,}\StringTok{"blue"}\NormalTok{), }
        \DataTypeTok{pch =} \KeywordTok{c}\NormalTok{(}\DecValTok{16}\NormalTok{, }\DecValTok{17}\NormalTok{, }\DecValTok{18}\NormalTok{))}
\end{Highlighting}
\end{Shaded}

\includegraphics{Hw1_files/figure-latex/unnamed-chunk-3-1.pdf}

\section{Problem 4}\label{problem-4}

\href{https://www.desmos.com/calculator/zv8fqgvvzt}{Here is a graph of
d1 and d2}

Both d1 and d2 take values of 0 when s is 1 and converge to infinity as
s approaches 0 (from the right). As such, both can be considered
dissimilarity measures on the interval {[}0, infinity). d2 results in
much large dissimilarity values.

\section{Problem 5}\label{problem-5}

\begin{enumerate}
\def\labelenumi{\alph{enumi})}
\item
  If the term only occurs in a single document, then the log term will
  be maximized, if the term occurs in every document then the log term
  will be 0
\item
  Common terms (the, a, because, that, etc.) that are likely to occur in
  most or all documents will have small values for tfij'. Uncommon words
  (diabetes, mahalanobis, dimensionality, etc.) will have larger values
  for tfij'. Words with a high tfij' might be of greater interest when
  trying to make predictions about a given document.
\end{enumerate}

\section{Problem 6}\label{problem-6}

\begin{enumerate}
\def\labelenumi{\alph{enumi})}
\tightlist
\item
  Hamming = 4 ; Jaccard = 2/6 = ⅓
\item
  SMC = 1 - (Hamming Distance) / (Number of Bits)
\item
  Both Jaccard and Cosine measures ignore 0-0 matches. For our example
  with binary vectors, the numerator for both measures is the number of
  1-1 matches.
\end{enumerate}

\section{Problem 7}\label{problem-7}

Cosine = (x * y) / sqrt(x * x) * sqrt(y * y) (* is dot product)
Correlation = covariance(x ,y) / (standard\_deviation(x) *
standard\_deviation(y)) Euclidian = sqrt(sum((x - y) \^{} 2)) (- and
\^{}2 are elementwise operations)

\begin{enumerate}
\def\labelenumi{\alph{enumi})}
\tightlist
\item
  Cosine = 0 ; Correlation = 0 ; Euclidian = 2
\item
  Cosine = 0 ; Correlation = -1 ; Euclidian = 2 ; Jaccard = 0
\end{enumerate}


\end{document}
